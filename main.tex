\documentclass[tombo,b5paper]{ltjsarticle}
\usepackage{bxpapersize}
\usepackage[luatex]{xcolor}
\usepackage[nosetpagesize, luatex]{graphicx}
\usepackage{pdfpages}
\usepackage{amsmath, amssymb}
\usepackage{epigraph}
\setlength{\epigraphwidth}{.85\textwidth}
\usepackage{luatexja-otf}
\usepackage{luatexja-fontspec}

\usepackage{url}
\usepackage{listings}
\usepackage{subfiles}
\usepackage[pdfencoding=auto,luatex]{hyperref}
\hypersetup{
  bookmarksnumbered=true,
  colorlinks=true,
  pdftitle={this is lualatex-ja testpdf},
  pdfauthor={hogehoge},
  pdfsubject={subtitle here}
}
\definecolor{myblue}{HTML}{268BD2}
\usepackage{tcolorbox}
\tcbuselibrary{breakable}
\newtcolorbox{tips}[1]{
  breakable,
  before skip=10pt plus 5pt minus 5pt,
  after skip=10pt plus 5pt minus 5pt,
  boxrule=0.3pt,
  colframe=myblue,
  colback=white!95,
  sharp corners=northwest,
  fonttitle=\gtfamily\bfseries,
  title=#1
}
\usepackage[colorinlistoftodos,prependcaption, textsize=tiny]{todonotes}
\usepackage[backend=biber,style=ieee]{biblatex}
\addbibresource{./library.bib}

% 単純にIPAexとかで良いなら、luatexja-presetを使うとよろしい
% \usepackage[ipa]{luatexja-preset}

% フォントは完全にテキトーに設定しているので、各位設定よろしく。
\setmainfont[Ligatures=TeX]{Source Serif Pro}
\setsansfont[Ligatures=TeX]{Source Sans Pro}

\setmainjfont[BoldFont=Source Han Sans Bold]{Source Han Sans ExtraLight}
\setsansjfont{Source Han Sans ExtraLight}

\newjfontfamily\jisninety[CJKShape=JIS1990]{Source Han Sans ExtraLight}

% わざとcolor周りは設定していない。自力でよろしく(solarizedとか良いよ)
\lstset{
  sensitive=true,
	breaklines=true,
	breakatwhitespace=true,
	framerule=0pt,
	frame=l
	showstringspaces=false,
	tabsize=2,
}

\author{@rizaudo}
\date{\today}
\title{同人誌の簡単テンプレート}

\begin{document}
\maketitle

\thispagestyle{empty}
\vspace*{\stretch{1}}
\begin{center}
\begin{minipage}{0.85\hsize}
  \begin{large}
    \textit{ここにサークルとかの文句を入れておくと良し}
    \begin{flushright}
      \textit{書いたやつの名前}
    \end{flushright}
  \end{large}
\end{minipage}
\end{center}
\vspace*{\stretch{1}}

\cleardoublepage


\section{はじめに}
…………ブウウ――――――ンンン――――――ンンンン………………。
 私がウスウスと眼を覚ました時、こうした蜜蜂の唸るような音は、まだ、その弾力の深い余韻を、私の耳の穴の中にハッキリと引き残していた。
 それをジッと聞いているうちに……今は真夜中だな……と直覚した。そうしてどこか近くでボンボン時計が鳴っているんだな……と思い思い、又もウトウトしているうちに、その蜜蜂のうなりのような余韻は、いつとなく次々に消え薄れて行って、そこいら中がヒッソリと静まり返ってしまった。
 私はフッと眼を開いた。

\subfile{sections/sample1}
\subfile{sections/sample2}


\section{おわりに}
ここに何か文を書く。
\begin{tips}{これがtips用のボックスです}
内容
\end{tips}
\todo[linecolor=red]{まだこの本は出来上がっていない。}
BibTeXを使ったciteもこのように\cite{Vinyals2015}出来る。
\listoftodos[TODOの一覧が出せるぞ]
\printbibliography[segment=\therefsegment, title=参考文献]
\end{document}